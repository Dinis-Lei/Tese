\chapter{Related Works}
\label{chapter:related_works}

\begin{introduction}
    Many authors have tackled the problem of estimating the energy consumption of parts of the \ac{ict}. However, very few have tried to estimate the energy consumption of the \ac{ict} as a whole. To make sure that the most appropriate models are used in this thesis, a review of the most recent studies was made. This review includes the criteria for selecting these models, a description on how the models work and the system boundary they act upon, and the pitfalls and assumptions of each model.
\end{introduction}

\section{Systematic review methodology}

The search methodology focused on studies that analyze energy or carbon consumption of parts of the \ac{ict} sector, such as datacenters. The search was based primarily on English language publications, using bibliographic databases such as IEEE Xplore Digital Library (\textbf{link}), ResearchGate (\textbf{link}). ACM Digital Library (\textbf{link}) and Google Scholar (\textbf{link}). Also, to complement the search we used ConnectedPapers (\textbf{link}), a tool that uses the references of a paper to create a graph of papers similar to the reference.

The search was based on primarily done by using a keyword related to the specific area of the \ac{ict} sector in study followed by a keyword related to energy consumption. For example a search for datacenters would be done by using the keywords \textit{"datacenter"} and \textit{"power consumption"}. The search filtered articles to include only those above January 2000. 
From the results gathered papers that showed bottom-up models were privileged over top-down models. 

\section{Energy models}

In this section, we will present the most relevant energy models gathered by the methodology described in the previous section. Each model tackles a different subsystem, however, their boundary delimitation can overlap with other models, so we will frame the boundaries of the models into the system boundaries that we have defined in \ref{section:system_boundaries}.

\subsection{Network models}

\subsection{Storage/Datacenter models}





% \subsection{Structure of the Internet}

% The internet is a network of networks, an infrastructure comprised of billions of computers, 
% using protocols,\ac{tcp}/\ac{ip} to communicate.

% It has always been a challenge to estimate the energy consumption of the internet, because of the complexe structure of the network. Nonetheless, several studies have been made to estimate the energy consumption. However, the results of the studies can go from 136 \ac{kilowatt-hour}/\ac{gigabyte}
% \citet{Koomey2003} down to 0.0064 \ac{kilowatt-hour}/\ac{gigabyte} \citet{Baliga2011}.

% As \citet{Coroama2015} and later \citet{Aslan2018} point out, the difference in results is due to the year that the study is made and the scope of the study. 
% The year is important because the energy efficiency of the equipment has been improving over the years. 
% The scope or system boundary is important because the studies can map different layers of the network.

% The internet is usually divided in layers as seen in the figure \ref{fig:layers}. 
% Each layer handles different amount on traffic so has different energy needs.
% The structure of the layers is discussed in the following sections.  

% \subsubsection{\acl{cpe} and Access network}

% In telecomunications, \ac{cpe} is the equipment that is installed in the end user premises that connects to the provider network. This usually means wifi routers and modems that connect to their \ac{isp} through a feeder network. The \ac{isp} bundles the data transmited by several end users in multiplexers, which for the case of \ac{isp} that use \ac{dsl} connectivety they are \ac{dslam}. The multiplexers and the cables that connect them to the \ac{cpe} are what constitute the access network.

% \subsubsection{Core network}

% Core network or IP core network is the network that interconnects several \ac{isp} with each other, forming the regional and global networks.
% The network consists of routers, switches, and other network infrastructure that directs data packets across the network. These devices use routing algorithms and protocols to determine the best path for data to travel based on factors like network congestion and available bandwidth.

% \subsection{Datacenters}

% Datacenters are facilities that house computer systems and associated components. They are a larges contributer to energy consumption in the internet, consuming a total of 205 \ac{terawatt-hour} in 2020, or 1\% of global eletric power \citet{IEA2020}. 