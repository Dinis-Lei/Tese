\chapter{Related Works}
\label{chapter:related_works}

\begin{introduction}
    Many authors have tackled the problem of estimating the energy consumption of parts of the \ac{ict}. However, very few have tried to estimate the energy consumption of the \ac{ict} as a whole. To make sure that the most appropriate models are used in this thesis, a review of the most recent studies was made. This review includes the criteria for selecting these models, a description on how the models work and the system boundary they act upon, and the pitfalls and assumptions of each model.
\end{introduction}

\section{Systematic review methodology}

The search methodology focused on studies that analyze energy or carbon consumption of parts of the \ac{ict} sector, such as datacenters. The search was based primarily on English language publications, using bibliographic databases such as IEEE Xplore Digital Library (\textbf{link}), ResearchGate (\textbf{link}). ACM Digital Library (\textbf{link}) and Google Scholar (\textbf{link}). Also, to complement the search we used ConnectedPapers (\textbf{link}), a tool that uses the references of a paper to create a graph of papers similar to the reference.

The search was based on primarily done by using a keyword related to the specific area of the \ac{ict} sector in study followed by a keyword related to energy consumption. For example a search for datacenters would be done by using the keywords \textit{"datacenter"} and \textit{"power consumption"}. The search filtered articles to include only those above January 2000. 
From the results gathered papers that showed bottom-up models were privileged over top-down models. 

\section{Energy consumption studies}

In this section, we will present the most relevant studies gathered by the methodology described in the previous section. Each study tackles a different subsystem, however, their boundary delimitation can overlap with others, so we will frame the boundaries of the studies into the system boundaries that we have defined in \ref{section:system_boundaries}.

The methodology used by each study is different. We separate the studies into those who use modeling, direct measurement, and those who use a combination of both.

\subsection{Modeling}

This approach consists in specifying equations based on parameters like energy consumption to describe the subsystem in study. The parameters can be obtained by using empirical data or by making informed assumptions of the system in question. 

\citet{Coroama2015} uses both approaches to estimate the energy consumption of the \ac{cpe} and access network subsystems. They developed a formula, (\ref{formula:coroama_cpe_an}), for estimating the energy intensity of these two subsystems. The formula builds on their previous analysis of multimedia servers \citet{Schien2013}. The formula is as follows: 

\begin{equation}
\label{formula:coroama_cpe_an}
    I_{CPE,AN} = \frac{t_{on}}{t_{use}} \frac{P_{CPE}}{N_{CPE}} + \frac{P_{AN}}{N_{AN}} PUE_{AN}.
\end{equation}

Where: 

\begin{itemize}
    \item $I_{CPE,AN}$: Intensity of both \ac{cpe} and \ac{an}.
    \item $\frac{t_{on}}{t_{use}}$: Ratio of time that the device is actively working. 
    \item $P_{CPE}$: Power of all \ac{cpe} devices.
    \item $N_{CPE}$: Number of Users connected to the \ac{cpe}.
    \item $P_{AN}$: Power of all \ac{an} devices.
    \item $N_{AN}$: Number of users (subscribers) connected to the \ac{an}.
    \item $PUE_{AN}$: \ac{pue} of the \ac{an}.
\end{itemize}

$\frac{P_{AN}}{N_{AN}}$ gives the energy intensity of the \ac{an} per subscriber, the technology they evaluate is \ac{adsl2} which has a power consumption of \SI{2}{\watt} per subscriber \citet{Schien2013}. As for the $PUE$ they assume a value of 2. 
$\frac{P_{CPE}}{N_{CPE}}$ gives the energy intensity of the \ac{cpe} per subscriber. The study assumes that each household has 2 \ac{cpe} devices, a modem and a router, and arrives at a value of \SI{8}{\watt} per subscriber when accounting for legacy equipment. 
The ratio of time that the device is actively working is estimated to be 6 according to the findings in \citet{Nissen2007}.
The study then concludes that the energy intensity for this subsystem is \SI{52}{\watt} per subscriber.

\citet{Schien2015} developed a model for the subsystem of core network. The study created a model to estimate the energy consumption of data traffic in the core network, that is the cost per \ac{gigabyte} of data transmitted. 
The model only considers the devices that will carry the connection data between the user and the server. Because not all energy spent by each device is directly related to a request, the energy intensity is the ratio between the energy spent by the device and the data throughput. The study distinguish between metro and long-haul networks, and use similar models for both. The model for the metro network is as follows:

\begin{equation}
\label{formula:schien_core_metro}
    I_{M_{TM}} = R \cdot n_{M_R} (c_{ON} \cdot I_{ON} + n_{M_{OA}} \cdot I_{OA})
\end{equation}

Where:

\begin{itemize}
    \item $R$: Redundancy.
    \item $n_{M_R}$: Number of metro routers.
    \item $c_{ON}$: Ratio of \ac{wdm} systems relative to routers.
    \item $I_{ON}$: Energy intensity per \ac{wdm} system.
    \item $n_{M_{OA}}$: Number of metro network optical amplifiers per hop.
    \item $I_{OA}$: Energy intensity per metro network optical amplifier.
\end{itemize}

%% Tall2014

Another model for the core network was developed by \citet{Tall2014}. The study models the energy consumption of offloading data between datacenters, so it includes in  its analysis the energy consumption of data being stored in datacenters as well as the cost of transmitting the data through the network. 
The study analysis the possibility of data to travel through public internet or by a dedicated network called \textit{lightpath}. The difference between the two is that the \textit{lightpath} does not need routing therefore the \ac{dwdm} nodes are directly connected to each other, whilst in the public internet the \ac{dwdm} nodes are connected to routers and switches. Figure \ref{fig:tall2014_network_components} shows the difference between the two networks. The energy consumption is defined as a function of data passing through the network ($D_{in}$). The models for the public internet and \textit{lightpath} are as follows:

\begin{equation}
\label{formula:tall_public_internet}
\begin{split}
    E_{internet}(D_{in}) = & \frac{PUE_{network}}{U} \cdot \frac{8D{in}}{3600} \cdot \\
    \Bigg( \bigg( \frac{2P_{switch}}{C_{switch}} + \frac{2P_{DWDM}}{C_{DWDM}} \bigg) + \bigg(\frac{2P_{switch}}{C_{switch}} & + \frac{2P_{DWDM}}{C_{DWDM}} + \frac{P_{router}}{C_{router}}\bigg) \cdot n_{hops} \Bigg)
\end{split}
\end{equation}

\begin{equation}
\label{formula:tall_lightpath}
\begin{split}
    E_{lightpath}(D_{in}) = & \frac{PUE_{network}}{U} \cdot \frac{8D{in}}{3600} \cdot \\
    \Bigg( \bigg( \frac{2P_{switch}}{C_{switch}} + \frac{2P_{DWDM}}{C_{DWDM}} \bigg) + \bigg(\frac{2P_{DWDM}}{C_{DWDM}}\bigg) \cdot n_{hops} & + \bigg( \frac{P_{switch}}{C_{switch}} \bigg) \cdot \bigg( n_{hops} - 1 \bigg) \Bigg)
\end{split}
\end{equation}

Where:

\begin{itemize}
    \item $PUE_{network}$: \ac{pue} of the network.
    \item $U$: Utilization of the network.
    \item $P_{switch}$: Power of the switch.
    \item $C_{switch}$: Capacity of the switch.
    \item $P_{DWDM}$: Power of the \ac{dwdm} node.
    \item $C_{DWDM}$: Capacity of the \ac{dwdm} node.
    \item $P_{router}$: Power of the router.
    \item $C_{router}$: Capacity of the router.
    \item $n_{hops}$: Number of hops between the datacenters.
\end{itemize}

As for the estimation of the energy consumption of the datacenter, the study considers 3 costs, the cost of writing data ($E_{write}$), reading data ($E_{read}$) and the cost of storing data ($E_{store}$). The study assumes a \ac{san} consisting of a content server, an Ethernet switch and a storage array of several disks. The equations for each cost are expressed below:

\begin{equation}
\label{formula:tall_datacenter_write}
    E_{write}(D_{in}) = \frac{PUE}{U} \cdot \frac{8D_{in}}{3600} \cdot \bigg(\frac{P_{server}}{C_{server}} + \frac{P_{sw}}{C_{sw}} + N_d(D_{in}) \frac{P_{disk}}{C_{disk}}  \bigg) 
\end{equation}

Where:
\begin{itemize}
    \item $PUE$: \ac{pue} of the datacenter.
    \item $U$: factor that accounts for the utilization of the data equipment, expressing the fact data equipment typically does not operate at a full utilization while still consuming 100\% of the power, $U = 0.5$.
    \item $\frac{8D_{in}}{3600}$: covertion of Gbps to GBph
    \item $P_x$: power consumption in kW of server, switch and disk.
    \item $C_x$: capacity of the server, switch and disk.
    \item $N_d(D_{in})$: function of the number of disks used for capacity $D_{in}$, assuming a configuration of RAID 10, given by: 
\end{itemize}

\begin{equation}
\label{formula:tall_datacenter_ndisks}
    N_d(D_{in}) = 2 \cdot \bigg \lceil \frac{D_{in}}{S_{array} \cdot S_{disk}} \bigg \rceil
\end{equation}

\begin{itemize}
    \item $S_{array}$: Number of disks in the array
    \item $S_{disk}$: Capacity of disks (GB)
\end{itemize}

\begin{equation}
\label{formula:tall_datacenter_store}
    E_{store}(D_{in}, RT) = \frac{PUE}{U} \cdot N_d(D_{in}) \cdot P_{disk} \cdot RT
\end{equation}

\begin{itemize}
    \item $RT$: retention time of the data stored
\end{itemize}

\begin{equation}
\label{formula:tall_datacenter_read}
    E_{read}(D_{out}) = \frac{PUE}{U} \cdot \frac{8D_{out}}{3600} \cdot \bigg(\frac{P_{server}}{C_{server}} + \frac{P_{sw}}{C_{sw}}  \bigg)
\end{equation}


\begin{itemize}
    \item $D_{out}$: Data being read.
\end{itemize}

%% Li Y

%% US Report 2016


\begin{center}
    \begin{tabular}{|| c | c | c ||}
        \hline
        Author & System Boundary & Reference \\
        \hline\hline
        1 & 2 & 3 \\
        \hline
    \end{tabular}
\end{center}


\subsection{Direct measurement}

%% Maldomin 2010


