

\chapter{Background}
\label{chapter:background}

\begin{introduction}
A short description of the chapter.

A memorable quote can also be used.
\end{introduction}

\section{Structure of Internet}

The internet is a network of networks, an infrastructure comprised of billions of computers, 
using protocols,\ac{tcp}/\ac{ip} to communicate. 
It is necessary to understand the different layers of the internet to correctly model the 
energy consumption of the network.

Several studies include end-users and datacenters in their models, when studying 
energy comsumption REFS, and while this study will include both we note that the 
separation of both from the network is important, because we aim to
construct a modular model that can remove any component depending on
the use case of whoever uses the model.

\citet{Coroama2015} and \citet{Schien2015} divide the network in 4 layers: \ac{cpe}, access, 
metro and core. Both studies use bottom-up approaches to calculate the energy consumption 
of each layer, creating models that use time used or data volume as idependent variables.
\citet{Coroama2015} focuses on the \ac{cpe} and access layers. They reach the conclusion that
in those layers, the energy consumption is dependent on the time that the equipment is used.
They research the power of the equipment involved and reach the conclusion that for 2015,
the energy intensity of the layers combined is 52 \ac{watts} per unit of time.



\section{Structure of Datacenters}

\section{Compression Algorithms}