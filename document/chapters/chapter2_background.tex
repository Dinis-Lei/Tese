

\chapter{Background}
\label{chapter:background}

\begin{introduction}
    
    The modelling of the energy consumption of the \ac{ict} sector is a complex task, beacuse it involves modelling several different systems, each with their distinct topologies and external factors.

    In this chapter we can find a descrition of the topology of the different systems that comprise the \ac{ict} sector and an highlight on the main component of each system. Moreover, the chapter also includes a description on the basics of compression algorithms and the different types of compression algorithms.

\end{introduction}

\section{System Boundaries}

The definition of system boundaries is an important factor in the modelling of energy consumption. As seen in the compilation of studies made by \citet{Aslan2018}, the system boundaries is one of the main factors for the discrepancies of the results.

The common consensus is that the system boundaries for the model of the internet should exclude datacenters and user devices \citet{Coroama2014}. This study however won't be modeling only the internet but also the impact on datacenters and of the compression/decompression algorithm in the end user device. So the overall system boundary will include all of the \ac{ict} sector. 

As will later be discussed in chapter \ref{chapter:energy_model}, the energy model can be divided in 3 main components, internet, datacenters and compression phase.
Swhich can be seen in figure \ref{fig:system_boundaries}.

\section{Internet}

The internet is the most complex component of the model.
Internet started has an interconnection of small number of local university networks, which were composed of a few routers and cables. Nowadays the internet is a global network of networks, composed of millions of routers, switches and cables, that interconnects billions of end users. This communication is made possible by the use of the \ac{tcp} and \ac{ip} protocols.

Because of its complexity the system is ususally divided in smaller components, however the delimitation on where each component begins and ends can vary from paper to paper. In this thesis we use the delimitation proposed by \citet{Coroama2015} and \citet{Schien2015}, which divides the internet in 4 main components: \ac{cpe}, access network, \ac{ip} core network and undersea cables.

\ref{fig:internet_topology} shows the topology of the internet as described by both papers \cite{Coroama2015} \cite{Schien2015}. 

\subsection{\acl{cpe} and Access Network}

There are several technologies used to establish to connect user to the Ethernet, they can be either wired or wireless and can be part of the \ac{cpe} layer or access network layer. Figure \ref{figure:network_technologies} highlights which technologies are currently used and in what layer they belong to.

\begin{figure}[h]
    \centering
    \includegraphics[width=0.8\textwidth]{figs/network_technologies.png}
    \caption[Network technologies used in Access Networks and CPE]{Network technologies used in Access Networks and CPE. In green are the technologies used in the access network and in blue are the technologies used in the CPE. Adapted from \citet{forum:huwawei}}
    \label{figure:network_technologies}
\end{figure}

The first layer and most simple is the \ac{cpe}. The \ac{cpe} is the equipment that is installed in the end user premises. 
The equipment is usually provided by the \ac{isp} and normaly consists of a gateway that acacts as a modem and a router.
A modem is used to convert the signal from the \ac{isp} to a signal that can be used by the router. The router is used to connect the end user devices to the internet. The user can connect to the routers either by cable or by wireless technology like Wi-fi. 

Access networks serve as the connection between the \ac{cpe} and the core network. All the user are connected to a central office where traffic is aggregated and then transfered to the core network.
The access network incorperates any technology that establishes the connection between the end user and the \ac{isp} and said technology it is not own by the user. % So a user can connect to the internet by either a modem or a cellular tower but only the cellular tower is part of the access network. 

Traffic in the access network is highly variable. The equipment used in the access network has a power consumption that is largely constant in time and thus load-independent \citet{Heddeghem2011}.

\subsubsection{Access Network Wired thecnologies}

The most popular wired connections are \ac{dsl}, coax cable and fiber-optics.

\begin{itemize}

    \item \ac{dsl} provide internet access over existing analog telephone lines. So, existing telephone service providers offer \ac{dsl} service. With this technology, user voice and data traffic go through this analog lines. \ac{dsl} uses high frequencies for data transmission. And with a help of \ac{dsl} filter data traffic do not interference with voice traffic.
    There are different DSL types. Now \ac{adsl} is the most used \ac{dsl} type in the world. 

    \item \ac{hfc} uses the same coaxial cable that is used for cable television. It uses the \ac{docsis} standard to provide internet access.

    \item Fiber-optic networks use light pulses transmitted through glass or plastic fibers, which produce high bandwidth and transmit speeds. 
    The fibers are also more resistant to interference and signal degradation, making them better suited for sending data over long distances without losing signal quality.
    Fiber optics technology, namely \ac{epon}, have become the most popular choice for wired access network technology. \ac{epon} is a point-to-multipoint network, which means that a single optical fiber is used to serve multiple end users. This is done by using a passive optical splitter, which splits the signal into multiple data streams that are sent to each end user.

\end{itemize}


\subsubsection{Access Network Wireless technologies}

Wireless access networks use radio waves to provide fixed or mobile access services for users. 
According to coverage range classification, the broadband wireless access technology are generally divided into different categories such as \ac{wpan}, a \ac{wlan}, a \ac{wman}, and a \ac{wwan}, however only technologies in the \ac{wman} and \ac{wwan} fall into the umbrella of what we refer as the access network.

The area coverage of \ac{wman} is in the order of kilometers and there are 2 prominent technologies for establish internet access, \ac{lmds} and \ac{wimax}.

\begin{itemize}

    \item \ac{lmds} as the name says is a multipoint communication system, it is used to provided network connectivity to buildings. These are great technologies for last mile connectivity, because they are more cost-effective than wired technologies and can be deployed faster than fiber \cite{forum:imda}.

    \item \ac{wimax}  is based on IEEE 802.16. It provides fixed, mobile or portible wireless connections with highspeeds that can reach up to 70 \ac{mbps} \cite{forum:ctrfantennasinc}.

\end{itemize}

The \ac{wwan} has a bigger range that the \ac{wman} and is used to provide internet access to mobile devices.
\ac{wwan} uses telecommunication cellular network technologies such as 2G, 3G, 4G LTE, and 5G to transfer data.


\subsection{Core Network}

Core networks consist of several core nodes that are interconnected by \ac{wdm} optical fiber links.

\ac{wdm} is a fiber-optic transmission technique that enables the use of multiple light wavelengths (or colors) to send data over the same medium. Two or more colors of light can travel on one fiber, and several signals can be transmitted in an optical waveguide at differing wavelengths or frequencies on the optical spectrum. 

Each core node is composed by a mix of several layers of technologies stacked on top of each other. Typicallty it is composed of a \ac{ip} layer, a \ac{wdm} layer and a \ac{oxc} layer.

From a broad perspective, core nodes operate on an optical-electrical-optical basis. This implies that any optical traffic undergoes conversion into the electronic domain and is then processed by the node, regardless of whether the traffic is terminated at that node. Typically, a node comprises several \ac{wdm} transmit and receive cards, commonly known as transponders or transceivers, which are linked to an \ac{ip} router. The \ac{ip} router, in turn, can establish connections with various access routers.

\section{Datacenters}

\subsection{Datacenter Network}

\subsection{Datacenter Servers}

\subsection{Datacenter Storage}

\section{Compression algorithms}
