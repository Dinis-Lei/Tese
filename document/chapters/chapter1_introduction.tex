

\chapter{Introduction}
\label{chapter:introduction}

\begin{introduction}

Energy consumption has been a topic of discussion lately due to the increase of global warming. The \ac{ict} sector is responsible for around 4-6\% of the global eletricity use in 2020 \citet{UK-parliament}. The main contributer to the energy consumption of the \ac{ict} sector are datacenters, which account for XXX\% (\citet{check source}). With the increase of internet use and the need to process large amounts of data, the energy consumption of the \ac{ict} sector will only increase further.

The primary objective of this thesis is to analyse data compression algorithms and their impact on the energy consumption of the \ac{ict} sector.
To achieve this objective, a model of the energy consumption of the \ac{ict} sector was develop, along side a web calculator to compare different compression algorithms on several uses cases. 

With the increase of machine learning applications and the need to process large amounts of data, the energy consumption of the \ac{ict} sector has been increasing. So models to estimate the energy consumption are needed to help policy makers and companies make informed decisions to meet the goals of Net Zero by 2050.

In this chapter, we will present reasearch questions, followed by the hypotheses and objectives of the thesis. We will then present the risk assessment and impact of the thesis, concluding with the contributions of the thesis. 

\end{introduction}

\section{Researh questions}
\label{section:research_questions}

To define the scope of the thesis, the following research questions were defined:

\begin{itemize}
    \item How to quantify the impact of compression algorithms on the energy consumption of the \ac{ict} sector?

    The impact can be quantified by firstly determening the energy intensity of data, that is the cost of processing a unif of data, ususally expressed in \ac{kilowatt-hour}/\ac{gigabyte}.
    The modeling of the energy consumption can be done by two approaches: bottom-up or top-down. 
    The bottom-up consists on aggregating data from each device of the system.
    The top-down approach uses the total energy consumption of the system and divides it by the amount of data processed.
    
    \item What is the appropriate system boundary of the model?
    
    The system boundary defines the scope of the model. It is important to define the system boundary to avoid over or under estimating the energy consumption of the \ac{ict} sector.

    \item How to evaluate the accuracy of the model?

    
    \item Which component will be more affected by the compression algorithms?

    Each component of the \ac{ict} sector is affected differently by the amount of data processed. Therefore, the impact of the compression algorithms will vary on each component.
\end{itemize}

\section{Hypotheses}
\label{section:hypotheses}

From the research questions, the following hypotheses were derived:

\begin{itemize}
    \item The energy consumption of the \ac{ict} sector can be modeled as a function of the data processed.

    By quantifying the compression ratio of each compression algorithm, the energy consumption of the \ac{ict} sector can be modeled as the differece between the uncompressed model and the compressed one.

    \item The most afected component affected by the compression algorithms is the datacenter.

    The datacenter is the component which consumes the most energy in the \ac{ict} sector. Therefore, it is expected that the datacenter will be the most affected component by the compression algorithms.

\end{itemize}


\section{Objectives}

To try to awnser the research questions posed in \ref{section:research_questions}, and to validate the hypotheses in chapter \ref{section:hypotheses}, the following objectives were defined:

\begin{itemize}
    \item Study the current energy models of the internet.

    \item Develop an aggregated model.

    \item Develop a web calculator to show the model

    \item Create case studies to benchmark the model

\end{itemize}

\section{Risk assessment and impact}

\section{Contributions}

The main contributions of this thesis are described below.
Firstly, an aggregated model of the energy consumption of the \ac{ict} was developed, which can be found in \ref{chapter:energy_model}.

Secondly, a web calculator was created to show the model. The web calculator allows the user to compare different compression algorithms on several use cases, which can be found in:

\begin{itemize}
    \item %\href{LINK}
\end{itemize}


\section{Thesis structure}

This thesis is divided into 6 chapters. After this introduction (chapter 1), a background on the concepts covered, such as network structure and compressions, is presented in chapter 2. 
Chapter 3 focuses on the recent works that have modeled part of the internet. 
An aggregation of these recent models will be presented in chapter 4. 
Chapter 5 will present the web calculator developed to show the model, along side with use cases on how to use the web calculator. 
Finally, chapter 6 will present the conclusions of the thesis and future work.


% - chapter 1: introduction
% - chapter 2: background
% - chapter 3: related work
% - chapter 4: Energy model
% - chapter 5: web calculator and use cases
% - chapter 6: conclusion