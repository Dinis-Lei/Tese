

\chapter{Introduction}
\label{chapter:introduction}

\begin{introduction}

Energy consumption has been a topic of discussion lately due to the increase of global warming. The \ac{ict} sector is responsible for around 4-6\% of the global eletricity use in 2020 \citet{UK-parliament}. The main contributer to the energy consumption of the \ac{ict} sector are datacenters, which account for XXX\% (\citet{check source}). With the increase of internet use and the need to process large amounts of data, the energy consumption of the \ac{ict} sector will only increase further.

The primary objective of this thesis is to analyse data compression algorithms and their impact on the energy consumption of the \ac{ict} sector.
To achieve this objective, a model of the energy consumption of the \ac{ict} sector was develop, along side a web calculator to compare different compression algorithms on several uses cases. 

In this chapter, we will present reasearch questions, followed by the hypotheses and objectives of the thesis. We will then present the risk assessment and impact of the thesis, concluding with the contributions of the thesis. 

\end{introduction}

\section{Researh questions}

To define the scope of the thesis, the following research questions were defined:

- What is the system boundary of the model?

- How to test the accuracy of the model?

- What is the impact of each component of the model?

\section{Hypotheses}

\section{Objectives}

- Study the current energy models of the internet

- Develop an aggregated model.

- Develop a web calculator to show the model

- Create case studies to benchmark the model


\section{Risk assessment and impact}

\section{Contributions}

\section{Thesis structure}

This thesis is divided into 6 chapters. After this introduction (chapter 1), a background on the concepts covered, such as network structure and compressions, is presented in chapter 2. Chapter 3 focuses on the recent works that have modeled part of the internet. An aggregation of these recent models will be presented in chapter 4. Chapter 5 will present the web calculator developed to show the model, along side with use cases on how to use the web calculator. Finally, chapter 6 will present the conclusions of the thesis and future work.


% - chapter 1: introduction
% - chapter 2: background
% - chapter 3: related work
% - chapter 4: Energy model
% - chapter 5: web calculator and use cases
% - chapter 7: conclusion